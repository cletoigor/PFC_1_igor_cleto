\addcontentsline{toc}{chapter}{Resumo}

\begin{center}
\huge{{\bf Resumo}}
\vspace{2cm}
\end{center}

%No Resumo, em uma única página, em no máximo dois parágrafos, você explicita os seguintes itens: objetivos do projeto e descrição sucinta do local onde ele foi desenvolvido; metodologia utilizada; e resultados alcançados. Leitores experientes decidem se prosseguirão para a leitura do texto completo após lerem o resumo, a conclusão e a introdução. Por isso nestes lugares você deve colocar um esforço maior de convencimento. Além disso, a linguagem utilizada deve ser acessível a leitores com pouca familiaridade com a área, limitando-se o uso de jargões.
%EDITAR FUTURAMENTE 

Este projeto tem como objetivo desenvolver um sistema inteligente para monitoramento e previsão do
consumo de energia em residências que não foram projetadas originalmente para automação residencial.
Especificamente, visa realizar uma pesquisa e análise de tecnologias existentes que permitam o
monitoramento energético sem a necessidade de infraestrutura prévia, explorar soluções adequadas para
residências convencionais, desenvolver um protótipo de sistema de monitoramento de fácil instalação e
uso, implementar algoritmos de processamento de dados e previsão de consumo utilizando técnicas de
inteligência artificial, criar uma plataforma de software intuitiva para visualização e análise dos dados pelos
usuários, testar e validar o sistema em ambiente real para garantir sua eficiência e usabilidade, promover a
viabilidade de implementação em larga escala por meio da análise de custos e estratégias de adoção, e
documentar e divulgar os resultados obtidos, destacando as contribuições para a Engenharia de Controle e
Automação e os benefícios sociais e ambientais proporcionados pelo sistema desenvolvido.

% \begin{sloppypar}
% Este novo parágrafo serve para mostrar que ao pular uma ou mais linhas no texto do arquivo .tex, o \TeX\ entende que você está iniciando outro parágrafo. O comando \textsf{sloppypar} força o texto a não ultrapassar as margens. Só deve ser usado se este problema ocorrer.
% \end{sloppypar}

\clearpage
\thispagestyle{empty}
\cleardoublepage

