\chapter{Metodologia}

Neste capítulo, são detalhadas as etapas metodológicas adotadas no desenvolvimento do sistema inteligente para monitoramento e previsão do consumo de energia residencial. A abordagem foi projetada para atender aos objetivos do projeto, combinando análise técnica, desenvolvimento de soluções e validação experimental. Os procedimentos descritos têm como objetivo garantir a reprodução, escalabilidade e eficácia do sistema proposto, contribuindo para o avanço na área de automação e eficiência energética.

\section{Pesquisa e Análise de Tecnologias Existentes}

A fase inicial do projeto consistiu na identificação e análise das tecnologias disponíveis no mercado para o monitoramento de consumo energético em residências. Esta etapa foi guiada por uma revisão bibliográfica abrangente e um estudo comparativo das soluções existentes, considerando os seguintes critérios:

\begin{itemize}
    \item \textbf{Compatibilidade com infraestruturas residenciais convencionais}: Tecnologias que não exigem modificações estruturais significativas, como sensores de fácil instalação e dispositivos autônomos.
    \item \textbf{Eficiência técnica}: Precisão na coleta de dados, robustez e confiabilidade das medições.
    \item \textbf{Interoperabilidade}: Capacidade de integração com diferentes sistemas e protocolos, como Wi-Fi, Zigbee e Bluetooth Low Energy.
    \item \textbf{Custo-benefício}: Viabilidade econômica para implementação em larga escala.
\end{itemize}

A análise resultou na seleção de componentes que atendem aos requisitos do projeto. Foi escolhida a linha de tomadas \textit{Tuya Smart Wi-Fi}, cuja modularidade permite fácil instalação em diversos locais, utilizando a metodologia \textit{plug and play}. Estas tomadas oferecem integração com a plataforma \textit{Tuya Cloud} e suas APIs, garantindo flexibilidade e escalabilidade.

\section{Desenvolvimento do Protótipo}

Com base nas tecnologias identificadas, foi desenvolvido um protótipo funcional que integra hardware e software para monitoramento e previsão de consumo energético. O desenvolvimento seguiu as seguintes etapas:

\subsection{Seleção e Configuração de Sensores}

As tomadas \textit{Tuya Smart Wi-Fi} foram configuradas para monitorar o consumo energético em tempo real. Estes dispositivos enviam os dados para a plataforma \textit{Tuya Cloud}, de onde são acessados por meio de APIs.

\subsection{Desenvolvimento do Sistema de Controle}

O sistema de controle foi desenvolvido utilizando as APIs fornecidas pela \textit{Tuya Cloud}, permitindo a comunicação entre as tomadas e o sistema central. O controlador é responsável por:

\begin{itemize}
    \item Receber e processar os dados coletados pelas tomadas.
    \item Aplicar algoritmos de análise e previsão de consumo.
    \item Integrar os dados a uma interface de visualização desenvolvida especificamente para o projeto.
\end{itemize}

\subsection{Desenvolvimento da Interface de Visualização}

A interface de monitoramento foi desenvolvida pelo autor do trabalho utilizando linguagens de programação voltadas para desenvolvimento web e visualização de dados. Esta interface exibe de forma intuitiva os padrões de consumo energético, permitindo que os usuários identifiquem desperdícios e ajustem seu comportamento para otimizar o uso de energia.

\section{Implementação de Algoritmos de Processamento e Previsão}

Para o tratamento dos dados coletados, foram desenvolvidos e implementados algoritmos específicos, detalhados a seguir:

\subsection{Pré-processamento dos Dados}

Os dados brutos coletados pelas tomadas passaram por etapas de limpeza e normalização, eliminando leituras espúrias e garantindo consistência nos resultados.

\subsection{Análise Estatística e Controle Estatístico de Processos (CEP)}

O CEP foi utilizado para identificar variações significativas no consumo energético, estabelecendo limites de controle e detectando anomalias. Esta abordagem foi complementada por métodos estatísticos descritivos para caracterizar os padrões de consumo.

\subsection{Previsão de Consumo com Inteligência Artificial}

Modelos de aprendizado de máquina foram treinados utilizando dados históricos para prever demandas futuras de energia. Técnicas como regressão linear e redes neurais foram aplicadas, visando maior precisão e adaptabilidade.

\section{Validação em Ambiente Real}

O protótipo foi testado em um ambiente residencial que não possuía infraestrutura prévia de automação. A validação envolveu:

\begin{itemize}
    \item \textbf{Avaliação de desempenho}: Testes para verificar a precisão das tomadas \textit{Tuya Smart Wi-Fi} e a eficiência dos algoritmos.
    \item \textbf{Usabilidade}: Análise da interface para garantir que usuários finais pudessem interpretar os dados facilmente.
    \item \textbf{Robustez}: Monitoramento contínuo em condições reais, avaliando a resiliência do sistema a falhas.
\end{itemize}

Os resultados dos testes foram documentados e utilizados para refinar o sistema, corrigindo inconsistências e melhorando sua eficiência.

\section{Viabilidade de Implementação em Larga Escala}

A análise de viabilidade considerou aspectos técnicos, econômicos e sociais, como:

\begin{itemize}
    \item \textbf{Custos de produção}: Avaliação dos investimentos necessários para replicação do sistema.
    \item \textbf{Benefícios econômicos}: Potencial de redução de custos para os usuários com base na diminuição do desperdício energético.
    \item \textbf{Impactos ambientais}: Contribuição para a sustentabilidade por meio do uso eficiente de energia.
    \item \textbf{Estratégias de adoção}: Identificação de barreiras e propostas para ampliar o alcance do sistema, como parcerias com fabricantes de dispositivos inteligentes.
\end{itemize}

\section{Resumo do Capítulo}

Este capítulo detalhou as etapas metodológicas seguidas no desenvolvimento do sistema inteligente para monitoramento e previsão de consumo de energia residencial. As etapas compreenderam a análise de tecnologias, desenvolvimento do protótipo utilizando as tomadas \textit{Tuya Smart Wi-Fi}, implementação de algoritmos avançados, validação experimental e estudo de viabilidade em larga escala. A abordagem proposta combina rigor técnico e aplicabilidade prática, destacando-se como uma solução promissora para o setor de automação residencial.